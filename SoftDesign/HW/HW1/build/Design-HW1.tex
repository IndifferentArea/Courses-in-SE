\documentclass[12pt, a4paper, oneside]{ctexart}
\usepackage{amsmath, amsthm, amssymb, appendix, bm, graphicx, mathrsfs, geometry, xcolor}
\geometry{left=2.54cm, right=2.54cm, top=3.18cm, bottom=3.18cm}
\usepackage[colorlinks, linkcolor=black]{hyperref}

\usepackage{listings}		% 为了避免与页眉的兼容问题可将listings放入table环境中
\lstset{
    basicstyle          =   \sffamily,          % 基本代码风格
    keywordstyle        =   \color{blue},          % 关键字风格
    keywordstyle    =   [2] \color{teal},
    commentstyle        =   \rmfamily\itshape,  % 注释的风格,斜体
    stringstyle         =   \ttfamily,  % 字符串风格
    flexiblecolumns,                % 别问为什么,加上这个
    numbers             =   left,   % 行号的位置在左边
    showspaces          =   false,  % 是否显示空格,显示了有点乱,所以不现实了
    numberstyle         =   \zihao{-5}\ttfamily,    % 行号的样式,小五号,tt等宽字体
    showstringspaces    =   false,
    captionpos          =   t,      % 这段代码的名字所呈现的位置,t指的是top上面
    frame               =   lrtb,   % 显示边框
    basicstyle          =   \zihao{-5}\ttfamily,
    stringstyle         =   \color{magenta},
    commentstyle        =   \color{red}\ttfamily,
    breaklines          =   true,   % 自动换行,建议不要写太长的行
    columns             =   fixed,  % 如果不加这一句,字间距就不固定,很丑,必须加
    basewidth           =   0.5em,
}


\linespread{1.5}
\renewcommand{\abstractname}{\Large\textbf{摘要}}

\begin{document}

\thispagestyle{empty}

\begin{figure}[t]
    \centering
    \includegraphics[width=13cm]{../pic/logo1.png}
\end{figure}

\vspace*{\fill}
    \begin{center}
        \Huge\textbf{Action \& ActionListener的实现\\及与其他事件处理机制的比较分析}
    \end{center}
\vspace*{\fill}

\begin{table}[b]
    \centering
    \large
    \begin{tabular}{ll}
        \textbf{课程:} & 软件系统设计与分析 \\
        \textbf{姓名:} & 杨豪 \\
        \textbf{班级:} & 软件2101 \\
        \textbf{时间:} & 2022年9月 \\
    \end{tabular}
\end{table}

\newpage

\thispagestyle{empty}
\begin{abstract}
    事件处理是面向对象分析与设计中最常用的功能之一. 本文先简单论述了事件、事件处理、事件处理机制的概念, 然后以Java语言分别实现了基本的事件处理机制: 
    同步调用、回调函数. 随后着重分析了Action\&ActionListener及其父类EventObject等接口的Java内部实现, 尝试用代码实现了一个简单的ActionListener机制. 
    最后总结了以上三种事件处理机制的优缺点. 
    \par\textbf{关键词:}事件处理机制; Java; 面向对象; Listener. 
\end{abstract}

\newpage
\pagenumbering{Roman}
\setcounter{page}{1}
\tableofcontents
\newpage
\setcounter{page}{1}
\pagenumbering{arabic}

\section{事件与事件处理机制概述}

\subsection{事件与事件处理}

    \begin{itemize}
        \item 事件(event): 在软件设计中, 事件是可由软件识别并处理的动作(action)或发生情形(occurrence), 通常来自外部环境, 可以由系统、用户或其他方式生成或触发. 
        \begin{itemize}
            \item 事件的来源可能是用户据通过计算机的外围设备与软件交互(eg.通过键盘输入); 
            \item 软件也可以触发它自己的事件集进入事件循环(eg.完成通信任务)
        \end{itemize}
        \item 事件处理(event processing): 软件识别某一特定事件进行并对其进行特定的处理方式
    \end{itemize}

\subsection{事件处理机制的组成}

    一个事件机制一般有三个组成部分, 这里以生活中的事为例: 我一边打游戏一边烧水
    \begin{itemize}
        \item 事件源(source): 即事件的发送者. 在上例中为水壶; 
        \item 事件(event): 事件源发出的一种信息或状态. 比如上例的警报声, 它代表着水开了; 
        \item 事件侦听者(listener): 对事件作出反应的对象. 比如上例中打游戏的我
    \end{itemize}
    在设计事件机制时一般把侦听者设计为一个函数, 当事件发送时, 调用此函数。比如上例中可以把倒水设计为侦听者. 

\subsection{简单的事件处理机制及其代码实现}

    \subsubsection{方法调用}

        方法调用即调用方\textbf{等待被调用方执行完毕并返回后}再继续执行, 因此是一种单向的阻塞式同步调用. 

        该方法用代码实现起来很简洁. 
        \lstinputlisting[
            language  =   Java,
            caption     =   {\bf method call.java},
            label       =   {eg1.py}
        ]{../src/eg1.java}

        输出结果如下
        \begin{lstlisting}[]
            // 第三行等了约5s后才输出
            This is Caller X, I will call Callee Y .
            This is Callee Y, I was called.
            Time cost is 5020 ms
        \end{lstlisting}

        显然, 因为\lstinline{callee}中的拖延, \lstinline{caller}也被拖了更多的时间. 

    \subsubsection{回调函数}

        回调函数: 回调是一种\textbf{双向}的调用模式, 设A和B是两个不同的类, 则B的该方法在被A调用, 完成后也会调用A的方法. 这样的调用需要定义一个interface, 
        从而将interface作为参数传入.

        这一思想实现起来比第一种略微麻烦一点: 
        \lstinputlisting[
            language  =   Java,
            caption     =   {\bf Callback.java},
            label       =   {eg2.py}
        ]{../src/eg2.java}
        输出结果如下
        \begin{lstlisting}[]
            // 第三行等了约5s后才输出
            Caller boss will assign Callee worker a task.
            This is Callee worker, I was called so I need to finish my task.
            Caller boss received a Notice: Callee worker has done his task.
        \end{lstlisting}
        尽管该方法似乎并没有改进什么, 但我们可以使用java的多线程编程, 在\lstinline{getcall}中插入如下的代码, 即以异步的形式实现, 
        则\lstinline{Caller}类就可以做自己的事而不必等待回复.
        \begin{lstlisting}[language = Java]
            new Thread(new Runnable()) {
                public void run() {
                    try {
                        sleep(5000);
                        cb.receiveNotice("Callee " + name + " has done his task.");
                    } catch (InterruptedException e) {
                        e.printStackTrace();
                    }
                }
            }
        \end{lstlisting}

\section{Action \& ActionListener机制分析与实现}

\subsection{Java中EventObject的基础概念}

    面向对象分析与设计中还有一种基本的事件处理机制, 即监听, 这种事件处理机制将事件的处理过程分成了四部分: 
    \begin{itemize}
        \item Source:事件源,即触发事件的对象, 任何具有行为的对象可以是事件源; 
        \item EventObject:事件对象, 即带有EventSource信息的事件对象, 也可以附带除source外的其它信息,是对EventSource的包装,  起到传递事件信息的作用; 
        \item EventListener:事件监听器,对该事件的处理, 一个监听器可以监视多个事件源; 
        \item 注册: 将Object和Listener绑定. 
    \end{itemize}

    \subsubsection{EventObject}

        \lstinline{EventObject}是定义于\lstinline{java.util}的一个类, 直接继承于Object, 是所有事件对象的父类. 
        \lstinline{ActionEvent}类的父类\lstinline{AWTEvent}也是继承于它(AWT是Java的一个用于编写图形应用程序的开发包). 

        EventObject提供了3个方法: 
        \begin{lstlisting}[language = Java]
            EventObject(Object source)  //source不能为null
            Object getSource();         //返回事件源
            String toString();          //返回该事件类名与事件源名    
        \end{lstlisting}

    \subsubsection{EventListener}

        EventListener也是定义于java.util包、直接继承于Object的接口, 是所有监听器的父接口, 无方法.

\subsection{Action \& ActionListener的原理}

经过查阅资料、源码和自己总结, Action\& ActionListener的实现原理如下
\begin{itemize}
    \item 定义过程
    \begin{enumerate}
        \item[a.] 为EventSource定义EventObject; 
        \begin{itemize}
            \item 组件( 如JFram、Button等) 触发Action等事件时EventObject的创建, 即定义过程的第一步, 是由JVM自动完成的. 
        \end{itemize}
        \item[b.] 为EventObject定义EventListener; 
        \begin{lstlisting}[language = java]
class MyActionListener implements ActionListener{
    @Override
    // 定义发生ActionEvent时要执行什么
    public void actionPerformed(ActionEvent e){
        // 一个Listener可能有多个Source, 所以通常先判断Source
        if(e.getSource() == myButtonOne){...}
        else if(e.getSource()==myButtonTwo{...}
        ...
        // if(e instanceof MyEvent)   
        // 不关注事件源, 而关注事件类型时, 通过事件类型判断
    }
}   
        \end{lstlisting}
        \item[c.] 定义EventSource的类,指定添加EventListener的方法. 
    \end{enumerate}
    \item 触发过程: 以Button组件为例
    \begin{enumerate}
        \item[a.] 通过 addActionListener方法, 为Button添加了ActionListener接口实现类; 
        \item[b.] Button类中定义的与点击相关的某个方法A 就会把自身作为source创建一个事件对象,并将该事件对象传入addActionListener方法所添加的监听器中; 
        \begin{lstlisting}[language = java]
JButton myButtonOne=new JButton("按钮一");
JButton myButtonTwo=new JButton("按钮二");
MyActionListener listener=new MyActionListener();
myButtonOne.addActionListener(listener);
myButtonTwo.addActionListener(listerner);
        \end{lstlisting}
        \item[c.] 在ActionListener中\textbf{根据source的类型或实例选择}执行某段代码(这里要用到前文中提到过的EventObject的getSource()方法). 
    \end{enumerate}
\end{itemize}

\subsection{Action \& ActionListener的代码实现}

    \lstinputlisting[
        language  =   Java,
        caption     =   {\bf Action.java},
        label       =   {eg3.py}
    ]{../src/eg3.java}
    输出结果如下, 代码通过\lstinline[language = Java]{implements EventListener}和\lstinline[language = Java]{extends EventObject}实现了监听器
    并具有可监听多个源的特性.
    \begin{lstlisting}[language = Java]
Alex: 今天天气不错
收到来自Alex的事件!
Bobby: 适合出去走走

当前共收到1次事件

Alex: 今天天气不错
收到来自Alex的事件!
Bobby: 适合出去走走
收到来自Bobby的事件!

当前共收到3次事件
    \end{lstlisting}


\section{事件处理机制的优劣比较}

\begin{itemize}
    \item 同步调用
    \begin{itemize}
        \item 优点:代码简洁,实现容易.
        \item 缺点: 调用者必须等待被调用者返回后再继续; 难以扩展新的方法.
    \end{itemize}
    \item 回调函数
    \begin{itemize}
        \item 优点: 代码相对容易实现; 可利用多线程编程, 调用者无需等待; 
        \item 缺点: 多线程控制复杂,调试麻烦; 
    \end{itemize}
    \item ActionListener
    \begin{itemize}
        \item 优点: 支持多线程编程; 代码逻辑比较简单, Java内有成熟的功能库; 工程库父子继承关系明确, 便于扩展新的方法
        \item 缺点: 比较依赖编程语言写好的框架, 如果完全自己实现会比较麻烦
    \end{itemize}
\end{itemize}

%%% reference
\newpage

\begin{thebibliography}{99}
    \bibitem{a}黄文海.Java多线程编程实战指南(设计模式篇).[M].北京:电子工业出版社,2006.
    \bibitem{b}action与actionlistener的关系.[DB/OL],\newline \url{https://zhuanlan.zhihu.com/p/65192541},2009.8.21.
    \bibitem{c}Java——事件处理机制.[DB/OL],
        \newline \url{https://blog.csdn.net/qq_19865749/article/details/70184968},2017.4.7.
    \bibitem{d}Java ActionListener Interface.[DB/OL].
        \newline \url{https://www.javatpoint.com/java-actionlistener},2021.
    \bibitem{e}Interface ActionListener java SE 17 \& JDK 17 [DB/OL]\newline
        \url{https://docs.oracle.com/en/java/javase/17/docs/api/java.desktop/java/awt/event/ActionListener.html},2022.
\end{thebibliography}


%%% Appendix
\newpage

\begin{appendices}
    \renewcommand{\thesection}{\Alph{section}}
    \section{\LaTeX 排版参考}
        本文中所用\LaTeX 代码参考自
        \begin{itemize}
            \item  \href{https://zhuanlan.zhihu.com/p/385727082}{【LaTeX】自用简洁模板(六):学校作业} 
            \item  \href{https://zhuanlan.zhihu.com/p/65441079}{LaTeX 里「添加程序代码」的完美解决方案}
        \end{itemize}
\end{appendices}



\end{document}