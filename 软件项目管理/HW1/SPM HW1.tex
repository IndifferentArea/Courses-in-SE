\documentclass[11pt]{article}  % need to compile twice
\usepackage{amsmath, textcomp, amssymb, geometry, graphicx, enumerate, ctex}
\usepackage[colorlinks, linkcolor=black]{hyperref}

\geometry{left=2.54cm, right=2.54cm, top=3.18cm, bottom=3.18cm}

\def\Name{杨豪\space}  % Your name
\def\SID{2206213297}  % Your student ID number

% need to be confirmed before each time writing and committing 
\def\Homework{1} % Number of Homework
\def\Session{2022-Fall}
\def\CourseCodeName{SOFT410511: 软件项目管理}
\def\simCourseName{SPM}

\title{\vspace{-4cm}\CourseCodeName \space
        \Session \protect\\  Homework-\textbf{\Homework} Solutions}
\author{软件2101 \Name \space 学号: \SID}
\markright{\simCourseName\ \space \Session\  HW-\Homework\ \Name}
\date{\today}

\begin{document}
\maketitle

\section*{练习题1. 以开篇案例为背景,用一两页纸的篇幅回答下述问题}

\section*{a. 你认为该案例中存在的真正问题是什么? }

案例中的真正问题是Nick没有找准自己的职业定位。

Nick的职位是项目经理而非技术研发,但他并没有专注于项目经理应该做的管理工作,
而把精力投入在技术研发的细节上,以至于从来没有为项日工作编制过准确的进度安排和详细计划并和高级管理层提交汇报。

作为项目经理,应该时刻牢记项目的干系人不仅仅有项目团队,还有上级领导。尽管Nick和项目团队的每个人都合作得很愉快,他并没有
满足他的上级领导的要求。

Nick虽然精力充沛、聪明过人,并带领团队按时交付了产品,但他显然没有意识到项目经理和软件开发员这两个职位的核心区别,依旧
在做其原来的“软件开发员”的工作。

\section*{b. 该案例是否反映了现实中的一个真实场景,为什么? }

该场景一定是真实存在的。

项目经理作为一个新兴的职业,大多数人甚至包括技术人员对其知之甚少。
由于项目经理这一岗位大多出现在大型的复杂项目中,而这样的项目往往又伴随大量技术人员,所以不少人将其与技术人员混为一谈,认为其只是技术人员的变体,
这样错误的认知才导致了案例中Nick的问题。这类问题由于技术人员大多追求性能和效果而非对项目整体的宏观管理,在现实生活中很常见。

\section*{c. Nick是不是一个好的项目经理,为什么? }

Nick不是一个好的项目经理。

他并没有理解到项目经理和技术开发这两个工作的本质区别,和原来的岗位一样把重点放在具体的技术细节攻克上,而忽视了项目综合管理和项目关系人。
项日管理要通过运用一定的知识、技能、工具和方法去满足项目需求,并同时满足或超越干系人的需求和期望。Nck并没有花时间
思考高级领导层对项目经理的期望和要求,他的认知局限在只要能够在预算范围内按时完成项目就足够了。

案例中提到,Nick的公司急于出成果,主要原因是为了满足正在与一家大公司进行收购谈判的需要,而生物技术项目也是该公司最大的项目,预期
将有巨大的增长潜力和潜在收益。因此,高级管理层需要尽可能实时地了解到DNA排序仪的进展和预期完成时间,预估项目难度,成本投入后续回报等等。
为了更完善地展现这些数据,需要向高级管理层项目管理计划书并不断作出修订。Nick并没有做到这些,而是仅仅关注技术实现和完成,直到项目结束。

这让高级管理层在项目研发过程中对项目的进度一无所知,不利于高级领导层在公司层面做出宏观规划。

\section*{d. 高层管理应怎样去帮助Nick? }

应该带领Nick完成一遍项目的具体流程,并出台一份关于项目经理的职责、工作要求和工作目的等常见问题的说明书便于后续如果出现类似情况
对其查阅。

项目具体流程包括:
\begin{enumerate}
    \item 制订项目章程
    \item 制订初步的项日范围说明书
    \item 开发项目管理计划
    \item 指挥并管理项日执行
    \item 实施项目管理计划
    \item 监控项目工作
    \item 实施综合变更控制,如果不符合要求需要回到第3步
    \item 结束项目
\end{enumerate}

\section*{e. 要成为一个好的项目经理,尼克应该怎么做}

\begin{enumerate}
    \item 对项目整体进行规划,明确项目经理的任务,形成项目综合计划,统筹协调管理项目所需的资源;
    \item 与项目干系人,包括客户、项目团队成员、高级领导层等进行沟通;
    \item 积极向高级管理层提交项目相关信息,听从高级管理层的指示
    \item 依照d问题中的指导学习执行项目经理的职责
\end{enumerate}

\section*{Reference}

    \LaTeX \space code refer to these things and was complied on texlive2020.
    \begin{itemize}
        \item  \href{https://www.eecs70.org/assets/misc/homework_template.tex}{UCB-CS70's given homework template.} 
        \item  \href{https://www.latexlive.com}{A free website useful to edit \LaTeX \space formula code.}
    \end{itemize}

    Thanks for your correcting and grading :).

\end{document}

 
