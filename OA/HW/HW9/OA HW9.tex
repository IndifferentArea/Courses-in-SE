\documentclass[11pt]{article}  % need to compile twice
\usepackage{amsmath, textcomp, amssymb, geometry, graphicx, enumerate, ctex, float, multirow}
\usepackage[colorlinks, linkcolor=black]{hyperref}
\usepackage[table,xcdraw]{xcolor}

\geometry{left=2.54cm, right=2.54cm, top=3.18cm, bottom=3.18cm}

\def\Name{杨豪\space}  % Your name
\def\SID{2206213297}  % Your student ID number

%%%%%%%%%%%%%%%%%%%%%%%%%%%%%%%%%%%%%%%%%%%%%%%%%%%%%%%%%%%%%%
% need to be confirmed before each time writing and committing 
\def\Homework{9} % Number of Homework
\def\Session{2022-Fall}
\def\CourseCodeName{SOFT400127: Computer Organization and Architecture}
\def\simCourseName{OA}

\title{\vspace{-4cm}\CourseCodeName \space
        \Session \protect\\  Homework-~\textbf{\Homework} Solutions}
\author{软件2101 \Name \space 学号: \SID}
\markright{\simCourseName\ \space \Session\  HW-\Homework\ \Name}
\date{\today}



\begin{document}
\maketitle

~\textbf{Honor Code: I promise that I finished the homework solutions on my own without copying other people's 
    work.}
    
\section*{Computer Arithmetic}

\textbf{Attention}: “|” is just for reading convenience.

\subsection*{1. }

Answer: \textbf{0|10000011|01001001100000000000000}, 

$ (20.59375)_{10} = (10100.10011)_2 = (1.010010011)_2 \times 2^4$

\begin{itemize}
    \item Sign. Positive float, 0. 
    \item Exponent. $4_{10} = (10000011)_2$(biased with -127).
    \item Mantissa. 010010011 along with following zeros.
\end{itemize}

Summary as below:

\begin{table}[H]
    \centering
    \begin{tabular}{|c|c|c|}
    \hline
    \cellcolor[HTML]{D2D2E7}\textbf{Sign (1 bit)} & \cellcolor[HTML]{C0DDC2}\textbf{Exponent (8 bits)} & \cellcolor[HTML]{DDD0C4}\textbf{Mantissa (23 bits)} \\ \hline
    \textbf{0}                                    & \textbf{10000011}                                  & \textbf{01001001100000000000000}                    \\ \hline
    \end{tabular}
\end{table}

\subsection*{2. }

Answer: \textbf{11.375}.

$(41360000)_{16} = (0100|0001|0011|0110|0000|0000|0000|0000)_2 = (0|10000010|01101100000000000000000)_2$
\begin{itemize}
    \item Sign. 0, so positive float.
    \item Exponent. $(10000010)_2\text{(biased with -127)}= 3$.
    \item Mantissa. Take off following zeros, $(1.011011)_2$
\end{itemize}
So the float is $(1.011011)_2\times 2^3 = (1011.011)_2 = 11.375$

\subsection*{3. }

\begin{table}[H]
    \centering
    \begin{tabular}{|c|c|c|c|c|c|}
    \hline
    Register & Variable & type                          & bits              & value         &                   \\ \hline
    R1       & x        & \multirow{2}{*}{unsigned int} & \textbf{10000110} & 134           &                   \\ \cline{1-2} \cline{4-6} 
    R2       & y        &                               & 11110110          & 246           &                   \\ \hline
    R3       & m        & \multirow{2}{*}{int}          & 10000110          & \textbf{-122} &                   \\ \cline{1-2} \cline{4-6} 
    R4       & n        &                               & 11110110          & -10           &                   \\ \hline
    R5       & z1       & \multirow{2}{*}{unsigned int} & \textbf{10010000} & 144           & overflow          \\ \cline{1-2} \cline{4-6} 
    R6       & z2       &                               & \textbf{01111100} & 124           & overflow          \\ \hline
    R7       & k1       & \multirow{2}{*}{int}          & 10000110          & \textbf{-112} &                   \\ \cline{1-2} \cline{4-6} 
    R8       & k2       &                               & 01111100          & 124           & \textbf{overflow} \\ \hline
    \end{tabular}
\end{table}

\section*{Other things}

\begin{itemize}
    \item \LaTeX \space code refer to these things and was complied on texlive2020. 
    \begin{itemize}
        \item  \href{https://www.eecs70.org/assets/misc/homework_template.tex}{UCB-CS70's given homework template.} 
        \item  \href{https://www.latexlive.com}{A free website useful to edit \LaTeX \space formula code.}
    \end{itemize}
    \item Some context refer to Professor Li.~'s PPT.
\end{itemize}
%% The purpose of writing in English is to adapt to bilingual teaching and to improve my poor English 
%% writing skills in preparation for a possible future exchange program. 

    Thanks for your correcting and grading :).

\end{document}