\documentclass[11pt]{article}  % need to compile twice
\usepackage{amsmath, textcomp, amssymb, geometry, graphicx, enumerate, ctex}
\usepackage[colorlinks, linkcolor=black]{hyperref}

\geometry{left=2.54cm, right=2.54cm, top=3.18cm, bottom=3.18cm}

\def\Name{杨豪\space}  % Your name
\def\SID{2206213297}  % Your student ID number

% need to be confirmed before each time writing and committing 
\def\Homework{1} % Number of Homework
\def\Session{2022-Fall}
\def\CourseCodeName{SOFT400127: Computer Organization \& Architecture}
\def\simCourseName{OA}

\title{\vspace{-4cm}\CourseCodeName \space
        \Session \protect\\  Homework-\textbf{\Homework} Solutions}
\author{软件2101 \Name \space 学号: \SID}
\markright{\simCourseName\ \space \Session\  HW-\Homework\ \Name}
\date{\today}





\begin{document}
\maketitle

Honor Code: I promise that I finished the homework solutions on my own without copying other people's 
    work.

\section*{Problem 2.10}

just to declare concept: 
\begin{enumerate}
    \item CPI: Clocks Per Instruction.
    \item MIPS: Million Instructions Per Second.
    \item Clock Frequency: Clocks per second(, which is 40MHz in this problem).
    \end{enumerate}

\subsection*{CPI}
    Answer: \textbf{1.55} 
    $$
    \begin{aligned}
        \text{CPI} = &\frac{\text{Total Clocks}}{\text{Instructions Count}} = 
            \frac{\sum\left(\text{Instruction Count} \times \text{Clocks per Instruction}\right)}
            {\sum\left(\text{Instruction Count}\right)}\\
        = & \frac{45000\times 1+32000 \times 2+15000\times 2+ 8000\times 2}{45000+32000+15000+8000} = 
            \frac{155000}{100000} = \frac{31}{20} = 1.55
    \end{aligned}
    $$

\subsection*{MIPS}
    Answer: \textbf{25.81Hz}
    $$
    \begin{aligned}
        \because \text{IPS} =& \frac{\text{Instructions Count}}{\text{Total Time}} = 
            \frac{\text{Instructions Count}}{\text{Total Clocks}}\cdot\frac{\text{Total Clocks}}
            {\text{Total Time}} = \frac{1}{\text{CPI}}\cdot \left(\text{Clock Frequency}\right)\\
        =&\frac{\text{Clock Frequency}}{\text{CPI}} = 
            \frac{40\times 10^6}{1.55} \approx 25.81 \times 10^6. \\
        \therefore \text{MIPS} =& 25.81
    \end{aligned}
    $$

\subsection*{execution time}
    Answer: \textbf{3.87 ms}
    $$
    \begin{aligned}
        \because & \text{IPS} = \frac{\text{Instructions Count}}{\text{Total Time}} \\
        \therefore &\text{Execution Time} = \text{Total Time} = \frac{\text{Instructions Count}}{\text{IPS}} = 
            \frac{100000}{25.81 \times 10^6} \approx 3.87\times 10^{-3}\text{s} = 3.87\text{ms}.
    \end{aligned}
    $$

\section*{Problem 2.12}
    Use VAX and IBM as short forms for VAX 11/780 and IBM RS/6000 respectively
\subsection*{a}
    Answer: As for count of machine code, 
    $\displaystyle \mathbf{\frac{\textbf{CISC}}{\textbf{RISC}} = \frac{2}{3}}$

    $$
    \begin{aligned}
        \because  &\text{IPS} = \frac{\text{Instructions Count}}{\text{Total Time}} 
        \therefore \text{Instructions Count} = \left(\text{Total Time}\right) \cdot \text{IPS}\\
        \therefore &\frac{\text{CISC Instructions Count}}{\text{RISC Instructions Count}} = 
            \frac{\left(\text{VAX Total Time}\right) \cdot \left(\text{VAX IPS}\right)}
            {\left(\text{IBM Total Time}\right) \cdot \left(\text{IBM IPS}\right)}
            = \frac{12x\cdot 1}{x \cdot 18} = \frac{2}{3} 
    \end{aligned}
    $$

\subsection*{b}
    Answer: $\displaystyle \textbf{VAX CPI} = 5, \textbf{IBM CPI} = \frac{25}{18} \approx 1.39$

    As evidenced above in 2.10-MIPS
    $$
    \begin{aligned}
        \because & \text{IPS} = \frac{\text{Clock Frequency}}{\text{CPI}}
        \therefore \text{CPI} = \frac{\text{Clock Frequency}}{\text{IPS}} \\ 
        \therefore &\text{VAX CPI} = \frac{5\times 10^6}{1\times 10^6} = 5, \quad
        \text{IBM CPI} = \frac{25\times 10^6}{18\times 10^6} = \frac{25}{18} \approx 1.39. 
    \end{aligned}
    $$

\section*{Other things}

\LaTeX \space code refer to these things and was complied on texlive2020.
\begin{enumerate}
    \item  UCB-CS70's given homework template 
    \item  \href{https://www.latexlive.com}{A free website useful to edit \LaTeX \space formula code.}
\end{enumerate}

The purpose of writing in English is to adapt to bilingual teaching and to improve my poor English 
writing skills in preparation for a possible future exchange program. 

Thanks for your correcting and grading :).

\end{document}

 
