\documentclass[11pt]{article}
\usepackage{amsmath, textcomp, amssymb, geometry, graphicx, enumerate, ctex, float}
\usepackage[colorlinks, linkcolor=black]{hyperref}
\usepackage{listings}		% 为了避免与页眉的兼容问题可将listings放入table环境中
\lstset{
    basicstyle          =   \sffamily,          % 基本代码风格
    keywordstyle        =   \color{blue},          % 关键字风格
    keywordstyle    =   [2] \color{teal},
    commentstyle        =   \rmfamily\itshape,  % 注释的风格,斜体
    stringstyle         =   \ttfamily,  % 字符串风格
    flexiblecolumns,                % 别问为什么,加上这个
    numbers             =   left,   % 行号的位置在左边
    showspaces          =   false,  % 是否显示空格,显示了有点乱,所以不现实了
    numberstyle         =   \zihao{-5}\ttfamily,    % 行号的样式,小五号,tt等宽字体
    showstringspaces    =   false,
    captionpos          =   t,      % 这段代码的名字所呈现的位置,t指的是top上面
    frame               =   lrtb,   % 显示边框
    basicstyle          =   \zihao{-4}\ttfamily,
    stringstyle         =   \color{magenta},
    commentstyle        =   \color{red}\ttfamily,
    breaklines          =   true,   % 自动换行,建议不要写太长的行
    columns             =   fixed,  % 如果不加这一句,字间距就不固定,很丑,必须加
    basewidth           =   0.5em,
}
\geometry{left=2.54cm, right=2.54cm, top=3.18cm, bottom=3.18cm}

\def\Name{杨豪\space}  % Your name
\def\SID{2206213297}  % Your student ID number

%%
% need to be confirmed before each time writing and committing 
\def\Homework{7} % Number of Homework
\def\Session{2022-Fall}
\def\CourseCodeName{SOFT400227: Operating System}
\def\simCourseName{OS}

\title{\vspace{-4cm}\CourseCodeName \space
        \Session \protect\\  Homework-\textbf{\Homework} Solutions}
\author{软件2101 \Name \space 学号: \SID}
\markright{\simCourseName\ \space \Session\  HW-\Homework\ \Name}
\date{\today}


\begin{document}
\maketitle
\vspace{-0.8cm}
\textbf{Honor Code: I promise that I finished the homework solutions on my own without copying other people's 
    work.}

\section*{Memory Management}

\subsection*{8.1}

\begin{table}[H]
    \centering
    \begin{tabular}{|c|c|c|}
    \hline
                 & Internal Fragmentation                                                                & External Fragmentation                                                               \\ \hline
    what         & \begin{tabular}[c]{@{}c@{}}allocated \\ but not actually used by program\end{tabular} & \begin{tabular}[c]{@{}c@{}}too short to store a program: \\ unallocated\end{tabular} \\ \hline
    reason       & fixed size allocation                                                                 & variable size allocation                                                             \\ \hline
    how to solve & dynamic memory blocks                                                                 & paging, segmentation                                                                 \\ \hline
    \end{tabular}
    \caption{Difference between Internal Fragmentation and External Fragmentation}
\end{table}

\subsection*{8.3}

Obviously, Best-fit is the most efficient algorithm.

\begin{table}[H]
    \centering
    \begin{tabular}{|c|c|c|c|}
        \hline
        memory partitions    & First-Fit                                                                              & Best-Fit             & Worst-Fit                                                                              \\ \hline
        100 KB               &                                                                                        &                      &                                                                                        \\ \hline
        500 KB               & \begin{tabular}[c]{@{}c@{}}212 KB (288 KB left) \\ $\rightarrow$ 112 KB (176 KB left)\end{tabular} & 417 KB (83 KB left)  & 417 KB (83 KB left)                                                                    \\ \hline
        200 KB               &                                                                                        & 112 KB (88 KB left)  &                                                                                        \\ \hline
        300 KB               &                                                                                        & 212 KB (88 KB left)  &                                                                                        \\ \hline
        600 KB               & 417 KB (183 KB left)                                                                   & 426 KB (174 KB left) & \begin{tabular}[c]{@{}c@{}}212 KB (388 KB left) \\ $\rightarrow$ 112 KB (276 KB left)\end{tabular} \\ \hline
        no enough memory for & 426 KB                                                                                 &                      & 426 KB                                                                                 \\ \hline
    \end{tabular}
    \caption{Allocation Result Chart}
\end{table}
\subsection*{8.9}

\subsubsection*{a. }

\textbf{Answer:} $2\times 200~\text{ns} = 400~\text{ns}$

It's required to access memory twice for paging, one to get physical address and another to get target memory.

\subsubsection*{b. }

\textbf{Answer:} $ 75\% \times 200~\text{ns}+ (1-75\%)\times 200~\text{ns}\times 2 = 250~\text{ns}$

When a physical is in TLB, only one memory access is need.

\subsection*{8.12}

\begin{table}[H]
    \centering
    \begin{tabular}{|c|c|c|c|c|c|}
        \hline
        & logical address & Segment & Base & Offset & physical address             \\ \hline
        a. & 0,430           & 0       & 219  & 430    & 649                          \\ \hline
        b. & 1,10            & 1       & 2300 & 10     & 2310                         \\ \hline
        c. & 2,500           & 2       & 90   & 500    & error (500\textgreater{}100) \\ \hline
        d. & 3,400           & 3       & 1327 & 400    & 1727                         \\ \hline
        e. & 4,112           & 4       & 1952 & 112     & error (112\textgreater{}96)  \\ \hline
        \end{tabular}
        \caption{physical addresses}
\end{table}

% they also need to be checked.
\section*{Other things}

    \LaTeX \space code refer to these things and was complied on texlive2020.
    \begin{itemize}
        \item  \href{https://www.eecs70.org/assets/misc/homework_template.tex}{UCB-CS70's given homework template.} 
        \item  \href{https://www.latexlive.com}{A free website useful to edit \LaTeX \space formula code.}
        \item  \href{https://www.tablesgenerator.com/}{A free website helpful to generate \LaTeX \space table.}
    \end{itemize}

    Some description refer to \textit{Operating System Concepts 10th}, \href{https://en.wikipedia.org}{Wikipedia} 
    and Professor.Tian's PPT.

    The purpose of writing in English is to adapt to bilingual teaching and to improve my poor English 
    writing skills in preparation for a possible future exchange program. 

    Thanks for your correcting and grading :).

\end{document}