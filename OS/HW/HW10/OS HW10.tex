\documentclass[11pt]{article}
\usepackage{amsmath, textcomp, amssymb, geometry, graphicx, enumerate, ctex, float}
\usepackage[colorlinks, linkcolor=black]{hyperref}
\usepackage{listings}		% 为了避免与页眉的兼容问题可将listings放入table环境中
\usepackage[normalem]{ulem}
\useunder{\uline}{\ul}{}
\lstset{
    basicstyle          =   \sffamily,          % 基本代码风格
    keywordstyle        =   \color{blue},          % 关键字风格
    keywordstyle    =   [2] \color{teal},
    commentstyle        =   \rmfamily\itshape,  % 注释的风格,斜体
    stringstyle         =   \ttfamily,  % 字符串风格
    flexiblecolumns,                % 别问为什么,加上这个
    numbers             =   left,   % 行号的位置在左边
    showspaces          =   false,  % 是否显示空格,显示了有点乱,所以不现实了
    numberstyle         =   \zihao{-5}\ttfamily,    % 行号的样式,小五号,tt等宽字体
    showstringspaces    =   false,
    captionpos          =   t,      % 这段代码的名字所呈现的位置,t指的是top上面
    frame               =   lrtb,   % 显示边框
    basicstyle          =   \zihao{-4}\ttfamily,
    stringstyle         =   \color{magenta},
    commentstyle        =   \color{red}\ttfamily,
    breaklines          =   true,   % 自动换行,建议不要写太长的行
    columns             =   fixed,  % 如果不加这一句,字间距就不固定,很丑,必须加
    basewidth           =   0.5em,
}
\geometry{left=2.54cm, right=2.54cm, top=3.18cm, bottom=3.18cm}

\def\Name{杨豪\space}  % Your name
\def\SID{2206213297}  % Your student ID number

%%
% need to be confirmed before each time writing and committing 
\def\Homework{10} % Number of Homework
\def\Session{2022-Fall}
\def\CourseCodeName{SOFT400227: Operating System}
\def\simCourseName{OS}

\title{\vspace{-4cm}\CourseCodeName \space
        \Session \protect\\  Homework-\textbf{\Homework} Solutions}
\author{软件2101 \Name \space 学号: \SID}
\markright{\simCourseName\ \space \Session\  HW-\Homework\ \Name}
\date{\today}


\begin{document}
\maketitle
\vspace{-0.8cm}
\textbf{Honor Code: I promise that I finished the homework solutions on my own without copying other people's 
    work.}

\section*{12.2}

% Please add the following required packages to your document preamble:

\begin{table}[H]
    \begin{tabular}{|c|c|c|c|c|c|c|c|c|c|c|c|c|c|c|}
    \hline
       &        &     &     &      &      &      &      &      &      &            &         &    &     & total distance \\ \hline
    a. & FCFS   & 143 & 86  & 1470 & 913  & 1774 & 948  & 1509 & 1022 & 1750       & 130     &    &     & 7081           \\ \hline
    b. & SSTF   & 143 & 130 & 86   & 913  & 948  & 1022 & 1470 & 1509 & 1750       & 1774    &    &     & 1745           \\ \hline
    c. & SCAN   & 143 & 913 & 948  & 1022 & 1470 & 1509 & 1750 & 1774 & {\ul 4999} & 130     & 86 &     & 9769           \\ \hline
    d. & LOOK   & 143 & 913 & 948  & 1022 & 1470 & 1509 & 1750 & 1774 & 130        & 86      &    &     & 3319           \\ \hline
    e. & C-SCAN & 143 & 913 & 948  & 1022 & 1470 & 1509 & 1750 & 1774 & {\ul 4999} & {\ul 0} & 86 & 130 & 9985           \\ \hline
    f. & C-LOOK & 143 & 913 & 948  & 1022 & 1470 & 1509 & 1750 & 1774 & 86         & 130     &    &     & 3363           \\ \hline
    \end{tabular}
\end{table}

\section*{I/O System}

\subsection*{1. }

设备独立性是指应用程序独立于具体使用的物理设备。

设备独立性概念的引入是为了实现设备分配时的灵活性和易于实现I/O重定向。

为了实现设备独立性引入逻辑设备和物理设备,应用程序使用逻辑设备名称来请求使用设备,系统需将逻辑设备名称转换为某物理设备。
因此也需要一个逻辑设备表LUT(Logical Unit Table) : 将应用程序中所使用的逻辑设备名映射为物理设备名。

\subsection*{2. }

SPOOLing技术是用于将一台独占设备改造成共享设备的一种行之有效的技术

利用多道程序中的一道程序来模拟脱机输入时的外围控制机的功能,把低速I/O设备上的数据传送到高速磁盘上;
用另一道程序来模拟脱机输出时外围控制机的功能,把数据从磁盘传送到低速输出设备上
从而实现脱机输入、输出功能,此时的外围操作与CPU对数据的处理同时进行。

\subsection*{3. }

\begin{itemize}
    \item 将接收到的抽象要求转换为具体要求。
    \item 检查用户 I/O 请求的合法性,了解 I/O 设备的状态,传递有关参数,设置设备的工作方式。
    \item 发出 I/O 命令,启动分配到的 I/O 设备,完成指定的 I/O 操作
    \item 及时响应由控制器或通道发来的中断请求,并根据其中断类型调用相应的中断处理程序进行处理。
    \item 对于设置有通道的计算机系统,驱动程序还应能够根据用户的 I/O 请求,自动地构成通道程序
\end{itemize}

% they also need to be checked.
\section*{Other things}

    \LaTeX \space code refer to these things and was complied on texlive2020.
    \begin{itemize}
        \item  \href{https://www.eecs70.org/assets/misc/homework_template.tex}{UCB-CS70's given homework template.} 
        \item  \href{https://www.latexlive.com}{A free website useful to edit \LaTeX \space formula code.}
        \item  \href{https://www.tablesgenerator.com/}{A free website helpful to generate \LaTeX \space table.}
    \end{itemize}

    Some description refer to \textit{Operating System Concepts 10th}, \href{https://en.wikipedia.org}{Wikipedia} 
    and Professor.Tian's PPT.

    Thanks for your correcting and grading :).

\end{document}