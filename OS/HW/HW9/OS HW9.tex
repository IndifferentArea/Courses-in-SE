\documentclass[11pt]{article}
\usepackage{amsmath, textcomp, amssymb, geometry, graphicx, enumerate, ctex, float}
\usepackage[colorlinks, linkcolor=black]{hyperref}
\usepackage{listings}		% 为了避免与页眉的兼容问题可将listings放入table环境中
\lstset{
    basicstyle          =   \sffamily,          % 基本代码风格
    keywordstyle        =   \color{blue},          % 关键字风格
    keywordstyle    =   [2] \color{teal},
    commentstyle        =   \rmfamily\itshape,  % 注释的风格,斜体
    stringstyle         =   \ttfamily,  % 字符串风格
    flexiblecolumns,                % 别问为什么,加上这个
    numbers             =   left,   % 行号的位置在左边
    showspaces          =   false,  % 是否显示空格,显示了有点乱,所以不现实了
    numberstyle         =   \zihao{-5}\ttfamily,    % 行号的样式,小五号,tt等宽字体
    showstringspaces    =   false,
    captionpos          =   t,      % 这段代码的名字所呈现的位置,t指的是top上面
    frame               =   lrtb,   % 显示边框
    basicstyle          =   \zihao{-4}\ttfamily,
    stringstyle         =   \color{magenta},
    commentstyle        =   \color{red}\ttfamily,
    breaklines          =   true,   % 自动换行,建议不要写太长的行
    columns             =   fixed,  % 如果不加这一句,字间距就不固定,很丑,必须加
    basewidth           =   0.5em,
}
\geometry{left=2.54cm, right=2.54cm, top=3.18cm, bottom=3.18cm}

\def\Name{杨豪\space}  % Your name
\def\SID{2206213297}  % Your student ID number

%%
% need to be confirmed before each time writing and committing 
\def\Homework{8} % Number of Homework
\def\Session{2022-Fall}
\def\CourseCodeName{SOFT400227: Operating System}
\def\simCourseName{OS}

\title{\vspace{-4cm}\CourseCodeName \space
        \Session \protect\\  Homework-\textbf{\Homework} Solutions}
\author{软件2101 \Name \space 学号: \SID}
\markright{\simCourseName\ \space \Session\  HW-\Homework\ \Name}
\date{\today}


\begin{document}
\maketitle
\vspace{-0.8cm}
\textbf{Honor Code: I promise that I finished the homework solutions on my own without copying other people's 
    work.}

\section*{File System}

\subsection*{10.1 }
 If the link still exists after the file it points to and a user try to access the file by it, he will access the wrong place.
 This illegal access will lead to serious result like change the wrong content.

 The way to solve the problem is when deleting the file, delete all link to it as well. Also, we can maintain a reference variable
 to record the link to the file so that if a link is deleted and the file has more than one link, the file system 
 only delete the link and decrease corresponding reference.

\subsection*{11.1}

Contiguous allocation scheme

\begin{table}[H]
    \centering
    \begin{tabular}{|c|c|c|}
    \hline
      & advantages                                                                                                          & disadvantages                      \\ \hline
    a & simple: a bit map/a free list                                                                                       & not flexible                       \\ \hline
    b & flexibility                                                                                                         & external fragmentation, complexity \\ \hline
    c & \begin{tabular}[c]{@{}c@{}}each size has a bit map/free list\\ intermediate complexity and flexibility\end{tabular} &                                    \\ \hline
    \end{tabular}
\end{table}

 
\subsection*{11.2}

Users can access the certain block of the file without loading the blocks before it to memory while it can still 
flexibly expand its size.

\subsection*{11.3}

\subsubsection*{a.}

4. 

Firstly get the block of root directory, secondly get the block of a (directory), thirdly get the block of b, 
eventually get the pointer to c.

\subsubsection*{b.}

Save the pointer in the disk and refresh after an interval of 8 fixed time.
 

% they also need to be checked.
\section*{Other things}

    \LaTeX \space code refer to these things and was complied on texlive2020.
    \begin{itemize}
        \item  \href{https://www.eecs70.org/assets/misc/homework_template.tex}{UCB-CS70's given homework template.} 
        \item  \href{https://www.latexlive.com}{A free website useful to edit \LaTeX \space formula code.}
        \item  \href{https://www.tablesgenerator.com/}{A free website helpful to generate \LaTeX \space table.}
    \end{itemize}

    Some description refer to \textit{Operating System Concepts 10th}, \href{https://en.wikipedia.org}{Wikipedia} 
    and Professor.Tian's PPT.

    The purpose of writing in English is to adapt to bilingual teaching and to improve my poor English 
    writing skills in preparation for a possible future exchange program. 

    Thanks for your correcting and grading :).

\end{document}