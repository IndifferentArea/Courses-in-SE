\documentclass[11pt]{article}
\usepackage{amsmath, textcomp, amssymb, geometry, graphicx, enumerate, ctex, float}
\usepackage[colorlinks, linkcolor=black]{hyperref}
\usepackage{listings}		% 为了避免与页眉的兼容问题可将listings放入table环境中
\lstset{
    basicstyle          =   \sffamily,          % 基本代码风格
    keywordstyle        =   \color{blue},          % 关键字风格
    keywordstyle    =   [2] \color{teal},
    commentstyle        =   \rmfamily\itshape,  % 注释的风格,斜体
    stringstyle         =   \ttfamily,  % 字符串风格
    flexiblecolumns,                % 别问为什么,加上这个
    numbers             =   left,   % 行号的位置在左边
    showspaces          =   false,  % 是否显示空格,显示了有点乱,所以不现实了
    numberstyle         =   \zihao{-5}\ttfamily,    % 行号的样式,小五号,tt等宽字体
    showstringspaces    =   false,
    captionpos          =   t,      % 这段代码的名字所呈现的位置,t指的是top上面
    frame               =   lrtb,   % 显示边框
    basicstyle          =   \zihao{-5}\ttfamily,
    stringstyle         =   \color{magenta},
    commentstyle        =   \color{red}\ttfamily,
    breaklines          =   true,   % 自动换行,建议不要写太长的行
    columns             =   fixed,  % 如果不加这一句,字间距就不固定,很丑,必须加
    basewidth           =   0.5em,
}
\geometry{left=2.54cm, right=2.54cm, top=3.18cm, bottom=3.18cm}

\def\Name{杨豪\space}  % Your name
\def\SID{2206213297}  % Your student ID number

% need to be confirmed before each time writing and committing 
\def\Homework{2} % Number of Homework
\def\Session{2022-Fall}
\def\CourseCodeName{SOFT400227: Operating System}
\def\simCourseName{OS}

\title{\vspace{-4cm}\CourseCodeName \space
        \Session \protect\\  Homework-\textbf{\Homework} Solutions}
\author{软件2101 \Name \space 学号: \SID}
\markright{\simCourseName\ \space \Session\  HW-\Homework\ \Name}
\date{\today}


\begin{document}
\maketitle

\textbf{Honor Code: I promise that I finished the homework solutions on my own without copying other people's 
    work.}

\section*{Process}

\subsection*{1. the differences among short-term, medium-term, and long-term scheduling}

    \begin{table}[H]
        \begin{tabular}{|l|l|l|l|}
            \hline
                & short-term scheduler        & medium-term scheduler & long-term scheduler            \\ \hline
            from   & ready, in memory            & blocked, in memory    &                                \\ \hline
            to     & be executed (allocated CPU) & disk                  & ready queue (in main memory)   \\ \hline
            other  &                             &                       & the degree of multiprogramming \\ \hline
            invoke & very frequently             &                       & very infrequently              \\ \hline
            atlas  & CPU scheduler               &                       & job scheduler                  \\ \hline
        \end{tabular}
    \end{table}

\subsection*{2. actions taken by a kernel to context-switch between processes}

    \begin{enumerate}
        \item The state of the currently executing process must be saved to PCB(Processor Control Block). PCB includes all the registers 
            that the process may be using, especially the program counter, plus any other operating system specific data 
            that may be necessary.
        \item The PCB might be stored on a stack in kernel memory. A handle to the PCB is added to the ready queue.
        \item When the process is rescheduled for execution, OS can then switch context by choosing a process from the ready queue and 
        restoring its PCB thus execution can continue in the chosen process.
    \end{enumerate}

\subsection*{3. the value of \lstinline{pid} \& \lstinline{pid1} in line A,B,C,D}

assume the pid of parent process and child process is 2600 and 2603 respectively.

    \begin{lstlisting}[language = C]
int main()
{
    pid_t pid, pid1;
    pid = fork();
    if (pid < 0){
        fprintf(stderr, "fork fail");
        return 1;
    }else if (pid == 0){
        pid1 = getpid();
        printf("child: pid=%d", pid);     //A
        printf("child: pid1=%d",pid1);    //B
    }else{
        pid1 = getpid();
        printf("parent: pid=%d",pid);     //C
        printf("parent: pid1=%d",pid1);   //D
        wait(NULL);
    }
    return 0;
}
    \end{lstlisting}

    Answer:
    \begin{table}[H]
        \centering
        \begin{tabular}{|c|c|c|}
            \hline
            A & pid  & \textbf{0}    \\ \hline
            B & pid1 & \textbf{2603} \\ \hline
            C & pid  & \textbf{2603} \\ \hline
            D & pid1 & \textbf{2600} \\ \hline
        \end{tabular}
    \end{table}

\subsection*{4. P104,3.10}

\begin{lstlisting}[language = C]
    CHILD: 0 CHILD: -1 CHILD: -4 CHILD: -9 CHILD: -16 PARENT: 0 PARENT: 1 PARENT: 2 PARENT: 3 PARENT: 4 
    // X line: printout num[] after modified in child process.
    // Y line: printout num[] in parent process. The array won't change with child process because they are seperate in different physical address after fork().
\end{lstlisting}

\section*{Other things}

    \LaTeX \space code refer to these things and was complied on texlive2020.
    \begin{itemize}
        \item  \href{https://www.eecs70.org/assets/misc/homework_template.tex}{UCB-CS70's given homework template.} 
        \item  \href{https://www.latexlive.com}{A free website useful to edit \LaTeX \space formula code.}
    \end{itemize}

    Some description refer to \textit{Operating System Concepts 10th}, \href{https://en.wikipedia.org}{Wikipedia} and Professor.Tian's PPT.

    The purpose of writing in English is to adapt to bilingual teaching and to improve my poor English 
    writing skills in preparation for a possible future exchange program. 

    Thanks for your correcting and grading :).

\end{document}