\documentclass[11pt]{article}
\usepackage{amsmath, textcomp, amssymb, geometry, graphicx, enumerate, ctex, float}
\usepackage[colorlinks, linkcolor=black]{hyperref}
\usepackage{listings}		% 为了避免与页眉的兼容问题可将listings放入table环境中
\lstset{
    basicstyle          =   \sffamily,          % 基本代码风格
    keywordstyle        =   \color{blue},          % 关键字风格
    keywordstyle    =   [2] \color{teal},
    commentstyle        =   \rmfamily\itshape,  % 注释的风格,斜体
    stringstyle         =   \ttfamily,  % 字符串风格
    flexiblecolumns,                % 别问为什么,加上这个
    numbers             =   left,   % 行号的位置在左边
    showspaces          =   false,  % 是否显示空格,显示了有点乱,所以不现实了
    numberstyle         =   \zihao{-5}\ttfamily,    % 行号的样式,小五号,tt等宽字体
    showstringspaces    =   false,
    captionpos          =   t,      % 这段代码的名字所呈现的位置,t指的是top上面
    frame               =   lrtb,   % 显示边框
    basicstyle          =   \zihao{-4}\ttfamily,
    stringstyle         =   \color{magenta},
    commentstyle        =   \color{red}\ttfamily,
    breaklines          =   true,   % 自动换行,建议不要写太长的行
    columns             =   fixed,  % 如果不加这一句,字间距就不固定,很丑,必须加
    basewidth           =   0.5em,
}
\geometry{left=2.54cm, right=2.54cm, top=3.18cm, bottom=3.18cm}

\def\Name{杨豪\space}  % Your name
\def\SID{2206213297}  % Your student ID number

% need to be confirmed before each time writing and committing 
\def\Homework{5} % Number of Homework
\def\Session{2022-Fall}
\def\CourseCodeName{SOFT400227: Operating System}
\def\simCourseName{OS}

\title{\vspace{-4cm}\CourseCodeName \space
        \Session \protect\\  Homework-\textbf{\Homework} Solutions}
\author{软件2101 \Name \space 学号: \SID}
\markright{\simCourseName\ \space \Session\  HW-\Homework\ \Name}
\date{\today}


\begin{document}
\maketitle
\vspace{-0.8cm}
\textbf{Honor Code: I promise that I finished the homework solutions on my own without copying other people's 
    work.}

\section*{Synchronous and Mutual Exclusion}

\subsection*{Problem 1. 6.4}
Because the spinlock requires busy waiting, so there have to be another process executing that waits for the signal about interrupt condition.

In single-processor systems, if the method's resource is the processor and doesn't release it, other processes won't be able to obtain 
interrupt signal.

In multiprocessor systems, processes can execute on other processors and obtain the signal so that the resources can be released.

\subsection*{Problem 2. 6.9}

If two same operations are executed on a semaphore at the same time and they
are not performed atomically, both operations might simultaneously operate on the semaphore value, which violates mutual exclusion. 

\subsection*{Problem 3. 6.11}

\begin{lstlisting}[language = Java]
Semaphore come_in = 1, customer = 0, baber = 1;
int available = N;

New_Customer{
    P(come_in);
    if(available != 0){
        V(customer);
        available --;
    }
    V(come_in);
}

Baber{
    P(customer);
    wake(baber);
    P(baber);
    while(++ available < N){
        P(customer);
    }
    V(baber);
    sleep(baber);
}
\end{lstlisting}


\subsection*{Problem 4. }
\begin{enumerate}
    \item Yes, easy to lead to starvation; 
    \item The mechanism of given abnormal PV operations is a stack but we need a queue. According to idea in Peterson's Algorithm, 
        we can set N 1-length stack to simulate queue as below:

\begin{lstlisting}[language = Java]
    Semaphore S[N - 1];
    for(int i = 0; i < N - 1; i ++) S[i] = N - i- 1;
	for(int i = 0; i < N - 1; i ++) P(S[i]);
	    // Critical Section
	for(int i = n - 2; i >= 0; i --) V(S[i]);
\end{lstlisting}

The algorithm above take O(n) space. 
Sadly, although this algorithm seems so easy and waste of characteristics, I still have no idea about how to optimize it.

However, it is said that it has proved that there is a method using only O(log(N)) space, amazing!
\end{enumerate}

\subsection*{Problem 5. }
We need 3 semaphore:
\begin{itemize}
    \item When there is a writer waiting, no more readers are allowed 
        so ~\lstinline{Writer()}~ should control this sequence by a semaphore which is a crucial part of ~\lstinline{Reader()}~ —— ~\lstinline{preRead}~.
        And this kind of control should be kept in a mutual exclusion environment, so we need another semaphore —— ~\lstinline{mutex}~.
    \item When there is a reader reading, writer should waiting until it finished 
    so as above, ~\lstinline{Reader()}~ need to guarantee this rule with a semaphore too. —— ~\lstinline{Writer}~.
    and in the ~\lstinline{mutex}~ also.
    \item There is always at most one writer writing at some time —— ~\lstinline{Writer}~ is adequate to deal with it.
\end{itemize}

\begin{lstlisting}[language = Java]
    Semaphore mutex = 1, write = 1, preRead = 1;
    int writeCount = 0, readCount = 0;
    
    Reader(){
        P(preRead);
        if(readCount==0) P(write);
        readCount++;
        V(preRead);

        // Reader Critical Section.

        P(mutex);
        readCount--;
        if(readCount==0) V(write);
        V(mutex);
    }
    
    Writer(){
        P(mutex);
        if(writeCount == 0) P(preRead);
        writeCount ++;
        V(mutex);
        P(write);

        // Writer Critical Section.

        P(mutex);
        writeCount --;
        if(writeCount == 0) V(preRead);
        V(write);
        V(mutex);
    }
\end{lstlisting}

\subsection*{Problem 6. }

We need 3 semaphore:
\begin{itemize}
    \item The door is only allowed for one person to get in and get out, so we need ~\lstinline{door}~ to deal with this mutual exclusion.
    \item There are N students inside the classroom, exam should begin, so we need ~\lstinline{exam}~ to deal with this synchronous.
    \item When the exam begin and the teacher handout papers, the door should be closed. After the teacher handout papers, 
        anybody who have handed can leave the classroom, so we need to open the door.We can use ~\lstinline{door}~ to control this flow. 
    \item When all papers handed in the teacher can leave the classroom and the exam ended.
        Students can hand in papers at the same time, so we just need one new semaphore, ~\lstinline{handin_done}~, to control this synchronous. 
\end{itemize}

\begin{lstlisting}[language = Java]
    Semaphore door = 1, exam = 0, handin_done = 0;
    int inCount = 0, paperCount = 0;
    In(){
        P(door);
        inCount ++;
        if(inCount == N) V(exam);
        V(door);
    }
    Out(){
        P(door);
        inCount --;
        V(door);
    }
    Exam(){
        P(exam);
        P(door);

        // Exam Critical Section: hand out papers.

        V(door);
        P(handin_done);
        // Teacher packaging papers;
        V(exam);
    }
    Handin(){
        paperCount ++;
        if(paperCount == N) V(handin_done);
    }
\end{lstlisting}


% they also need to be checked.
\section*{Other things}

    \LaTeX \space code refer to these things and was complied on texlive2020.
    \begin{itemize}
        \item  \href{https://www.eecs70.org/assets/misc/homework_template.tex}{UCB-CS70's given homework template.} 
        \item  \href{https://www.latexlive.com}{A free website useful to edit \LaTeX \space formula code.}
        \item  \href{https://www.tablesgenerator.com/}{A free website helpful to generate \LaTeX \space table.}
    \end{itemize}

    Some description refer to \textit{Operating System Concepts 10th}, \href{https://en.wikipedia.org}{Wikipedia} and Professor.Tian's PPT.

    The purpose of writing in English is to adapt to bilingual teaching and to improve my poor English 
    writing skills in preparation for a possible future exchange program. 

    Thanks for your correcting and grading :).

\end{document}