\documentclass[11pt]{article}
\usepackage{amsmath, textcomp, amssymb, geometry, graphicx, enumerate, ctex, float}
\usepackage[colorlinks, linkcolor=black]{hyperref}
\usepackage{listings}		% 为了避免与页眉的兼容问题可将listings放入table环境中
\lstset{
    basicstyle          =   \sffamily,          % 基本代码风格
    keywordstyle        =   \color{blue},          % 关键字风格
    keywordstyle    =   [2] \color{teal},
    commentstyle        =   \rmfamily\itshape,  % 注释的风格,斜体
    stringstyle         =   \ttfamily,  % 字符串风格
    flexiblecolumns,                % 别问为什么,加上这个
    numbers             =   left,   % 行号的位置在左边
    showspaces          =   false,  % 是否显示空格,显示了有点乱,所以不现实了
    numberstyle         =   \zihao{-5}\ttfamily,    % 行号的样式,小五号,tt等宽字体
    showstringspaces    =   false,
    captionpos          =   t,      % 这段代码的名字所呈现的位置,t指的是top上面
    frame               =   lrtb,   % 显示边框
    basicstyle          =   \zihao{-4}\ttfamily,
    stringstyle         =   \color{magenta},
    commentstyle        =   \color{red}\ttfamily,
    breaklines          =   true,   % 自动换行,建议不要写太长的行
    columns             =   fixed,  % 如果不加这一句,字间距就不固定,很丑,必须加
    basewidth           =   0.5em,
}
\geometry{left=2.54cm, right=2.54cm, top=3.18cm, bottom=3.18cm}

\def\Name{杨豪\space}  % Your name
\def\SID{2206213297}  % Your student ID number

%%
% need to be confirmed before each time writing and committing 
\def\Homework{8} % Number of Homework
\def\Session{2022-Fall}
\def\CourseCodeName{SOFT400227: Operating System}
\def\simCourseName{OS}

\title{\vspace{-4cm}\CourseCodeName \space
        \Session \protect\\  Homework-\textbf{\Homework} Solutions}
\author{软件2101 \Name \space 学号: \SID}
\markright{\simCourseName\ \space \Session\  HW-\Homework\ \Name}
\date{\today}


\begin{document}
\maketitle
\vspace{-0.8cm}
\textbf{Honor Code: I promise that I finished the homework solutions on my own without copying other people's 
    work.}

\section*{Memory Management}

\subsection*{9.4}

virtual address: $11123456_{16}={0001 0001 0001 0010 0011 0100 0101 0110}_2$

virtual-memory has $\displaystyle \frac{2^{32}}{4096} = 2^{20}$ pages.

physical-memory has $\displaystyle \frac{2^{18}}{4096} = 2^6$ pages.

So the first 20 bits of virtual address should be translated into 6 bits by page-table and the 6 bits result 
should combine with other 12 bits to generate physical address.

Page tables are in hardware, but a little software is needed to tell CPU to use it.

\subsection*{9.10}

The system takes too much memory of paging disk. If the degree of multiprogramming decreases or the physical memory is improved
the process will page fault less frequently. 

\begin{enumerate}[a. ]
    \item No.
    \item No.
    \item No. Contrary effect.
    \item Yes.
    \item Yes if the memory is big enough.
    \item Yes. The same reason as e.
    \item Yes since CPU can transfer data faster.
    \item Yes since page fault will be less and more data will be transferred once.
\end{enumerate}

\subsection*{9.13}

\subsubsection*{a.}

\begin{enumerate}
    \item initial value is 0;
    \item when the page is accessed, add by 1;
    \item when the page is replaced, decreased to 1;
    \item choose the smallest counter, and if number is equal, FIFO.
\end{enumerate}

\subsection*{b.}
13
\begin{table}[H]
    \begin{tabular}{|c|c|c|c|c|c|c|c|c|c|c|c|c|c|c|c|c|c|c|}
    \hline
    1-4          & \textbf{5}   & \textbf{3}   & 4            & \textbf{1}   & \textbf{6}   & \textbf{7}   & \textbf{8}   & \textbf{7}   & \textbf{8}   & \textbf{9}   & \textbf{7}   & \textbf{8}   & \textbf{9}   & \textbf{5}   & \textbf{4}   & \textbf{5}   & \textbf{4}   & \textbf{2}   \\ \hline
    \textbf{1 1} & \textbf{5 1} & 5 1          & 5 1          & 5 1          & \textbf{6 1} & 6 1          & 6 1          & 6 1          & 6 1          & \textbf{9 1} & 9 1          & 9 1          & \textbf{9 2} & 9 2          & \textbf{4 1} & 4 1          & \textbf{4 2} & 4 2          \\ \hline
    \textbf{2 1} & 2 1          & 2 1          & 2 1          & \textbf{1 1} & 1 1          & \textbf{7 1} & 7 1          & \textbf{7 2} & 7 2          & 7 2          & \textbf{7 3} & 7 3          & 7 3          & 7 3          & 7 3          & 7 3          & 7 3          & \textbf{2 1} \\ \hline
    \textbf{3 1} & 3 1          & \textbf{3 2} & 3 2          & 3 2          & 3 2          & 3 2          & \textbf{8 1} & 8 1          & \textbf{8 2} & 8 2          & 8 2          & \textbf{8 3} & 8 3          & 8 3          & 8 3          & 8 3          & 8 3          & 8 3          \\ \hline
    \textbf{4 1} & 4 1          & 4 1          & \textbf{4 2} & 4 2          & 4 2          & 4 2          & 4 2          & 4 2          & 4 2          & 4 2          & 4 2          & 4 2          & 4 2          & \textbf{5 1} & 5 1          & \textbf{5 2} & 5 2          & 5 2          \\ \hline
    \end{tabular}
\end{table}

\subsection*{c.}
11
\begin{table}[H]
    \centering
    \begin{tabular}{|c|c|c|c|c|c|c|c|c|c|c|c|c|c|c|c|c|c|c|}
    \hline
    1-4 & \textbf{5} & \textbf{3} & 4         & \textbf{1} & \textbf{6} & \textbf{7} & \textbf{8} & \textbf{7} & \textbf{8} & \textbf{9} & \textbf{7} & \textbf{8} & \textbf{9} & \textbf{5} & \textbf{4} & \textbf{5} & \textbf{4} & \textbf{2} \\ \hline
    1   & \textbf{}  &            &           &            & 6          &            & 8          &            &            & \textbf{}  &            &            & \textbf{}  &            & \textbf{}  &            & \textbf{}  & 2          \\ \hline
    2   & 5          &            &           & \textbf{}  &            & \textbf{}  &            & \textbf{}  &            &            & \textbf{}  &            &            &            &            &            &            & \textbf{}  \\ \hline
    3   &            & \textbf{}  &           &            &            & 7          & \textbf{}  &            & \textbf{}  &            &            & \textbf{}  &            &            & 4          &            &            &            \\ \hline
    4   &            &            & \textbf{} &            &            &            &            &            &            & 9          &            &            &            & \textbf{}  &            & \textbf{}  &            &            \\ \hline
    \end{tabular}
\end{table}

% they also need to be checked.
\section*{Other things}

    \LaTeX \space code refer to these things and was complied on texlive2020.
    \begin{itemize}
        \item  \href{https://www.eecs70.org/assets/misc/homework_template.tex}{UCB-CS70's given homework template.} 
        \item  \href{https://www.latexlive.com}{A free website useful to edit \LaTeX \space formula code.}
        \item  \href{https://www.tablesgenerator.com/}{A free website helpful to generate \LaTeX \space table.}
    \end{itemize}

    Some description refer to \textit{Operating System Concepts 10th}, \href{https://en.wikipedia.org}{Wikipedia} 
    and Professor.Tian's PPT.

    The purpose of writing in English is to adapt to bilingual teaching and to improve my poor English 
    writing skills in preparation for a possible future exchange program. 

    Thanks for your correcting and grading :).

\end{document}