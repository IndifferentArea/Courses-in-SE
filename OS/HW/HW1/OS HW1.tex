\documentclass[11pt]{article}
\usepackage{amsmath, textcomp, amssymb, geometry, graphicx, enumerate, ctex}
\usepackage[colorlinks, linkcolor=black]{hyperref}

\geometry{left=2.54cm, right=2.54cm, top=3.18cm, bottom=3.18cm}

\def\Name{杨豪\space}  % Your name
\def\SID{2206213297}  % Your student ID number

% need to be confirmed before each time writing and committing 
\def\Homework{1} % Number of Homework
\def\Session{2022-Fall}
\def\CourseCodeName{SOFT400227: Operating System}
\def\simCourseName{OS}

\title{\vspace{-4cm}\CourseCodeName \space
        \Session \protect\\  Homework-\textbf{\Homework} Solutions}
\author{软件2101 \Name \space 学号: \SID}
\markright{\simCourseName\ \space \Session\  HW-\Homework\ \Name}
\date{\today}





\begin{document}
\maketitle

\textbf{Honor Code: I promise that I finished the homework solutions on my own without copying other people's 
    work.}

\section*{Part-1 Introduction}

\subsection*{1.1 main advantage of multiprogramming}

It \textbf{improves utilization of CPU} and other hardware resources by keeping serval processes 
    in memory simultaneously instead of just one. When the current process needs to wait for
    some tasks, the CPU switch to the next process and so on.

\subsection*{1.2 Systems' Concepts}

    \begin{enumerate}[a.]
        \item \textbf{Batch Systems} process jobs in bulk, with predetermined input.
            \begin{itemize}
                \item \textbf{Simple Batch Systems} Only execute a job \textbf{until it is finished} and switch to the next job. 
                \item \textbf{Multi-programmed Batch Systems}: When the current job needs to wait for some tasks(resources), the CPU 
                    switch to the next job and so on. After waiting job get what it needs, it returns to wait for CPU.
            \end{itemize}
        \item \textbf{Time sharing Systems} use a timer and scheduling algorithms to \textbf{cycle processes rapidly} through CPU, giving each 
            user a sense of sharing the resources with each other.
        \item \textbf{Real time Systems} have \textbf{well-defined, fixed time constraints}, which processing must be done within 
        otherwise it will fail. This kind of highly serious constraints is suitable in where rigid real-time tasks is required 
        such as weapon systems, industrial control systems and so on.
        \item \textbf{Network Systems} provides \textbf{communication} mechanisms between computers comparing to normal operating systems.
            Unlike distributed systems, computers have individual threads and processes.
        \item \textbf{Distributed Systems} is a collection of independent computers that appears to its user as a single coherent 
            system, which means it should provide user transparency.
            More clearly speaking, in hardware aspect, computers are individuals, while they are a entire computer logically.
    \end{enumerate}

\subsection*{1.3}
    I will use \textbf{Y}/\textbf{N} to signify whether an instruction is privileged instruction or not.
\begin{enumerate}[A.]
    \item Set the Timer: \textbf{Y}. User mode can't set value otherwise malicious code may make the kernel mode interrupt
        at the wrong time.
    \item Read the Clock: \textbf{N}. Just reading status is secure for computer.
    \item Clear the Memory: \textbf{Y}. It could clear the kernel stack in the memory even kill a process without kernel mode. 
    \item Turn off the Interrupt: \textbf{Y}. Without interrupt malicious code won't be detected.
    \item Switch from user mode to monitor mode(aka. kernel mode): \textbf{N}. eg. trap().
\end{enumerate}

\section*{Part-2 OS structures}
   
\subsection*{2.1}
To provide interfaces between running program and operating system so that a wall will isolate user mode from kernel mode. 

\subsection*{2.2}
The main advantage of the layered approach is simplicity of construction and debugging. When facing a bug or unclear error, 
the bug can be found by debugging from lowest-level layer to highest level layer. With such design, user can just focus on certain
layer when debugging.

\subsection*{2.3}
The kernel is micro and well designed so it is easy to extend the operating system and to be ported from one hardware to another.

eg. MacOS's core Darwin is a microkernel OS, so MacOS can be easily ported from Intel to ARM
(Apple's own M-series chips are designed based on ARM architecture).

\section*{Other things}

\LaTeX \space code refer to these things and was complied on texlive2020.
\begin{itemize}
    \item  \href{https://www.eecs70.org/assets/misc/homework_template.tex}{UCB-CS70's given homework template.} 
    \item  \href{https://www.latexlive.com}{A free website useful to edit \LaTeX \space formula code.}
\end{itemize}

Some description refer to \textit{Operating System Concepts 10th}, \href{https://en.wikipedia.org}{Wikipedia} and Professor.Li's PPT.

The purpose of writing in English is to adapt to bilingual teaching and to improve my poor English 
writing skills in preparation for a possible future exchange program. 

Thanks for your correcting and grading :).

\end{document}