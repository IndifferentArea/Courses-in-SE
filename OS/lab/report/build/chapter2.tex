\vspace{-3cm}\chapter{实验2:Linux文件管理}

\section{2.1 实验目的}

% 熟练掌握Linux操作系统的使用,掌握Linux的系统的进程管理和文件管理功能。
% 实验要求:
% 完成实验内容并写出实验报告,报告应具有以下内容:
% 1)	实验目的;
% 2)	实验内容;
% 3)	题目分析及基本设计过程分析;
% 4)	配置文件关键修改处的说明及运行情况,应有必要的效果截图;
% 5)	实验过程中出现的问题及解决方法;
% 6)	实验体会。

\section{2.2 实验内容}

\begin{enumerate}
    \item 将若干已有用户加入到同一个组xjtuse中。在/home下创建一个共享的公用目录public,
        允许xjtuse组中的用户对该目录具有读写和执行操作。(给出相关命令及运行结果)
    \item 对于public目录下的文件,只有文件的拥有者才具有删除文件的权限。(给出相关命令及运行结果)
    \item 对于public目录下的文件,也可以通过路径/mnt/public来访问。(给出相关命令及运行结果)
    \item 看Linux系统磁盘空间的使用情况(给出显示结果),并为/分区创建磁盘配额,
        使得用户可用空间的软限制为100M,硬限制为150M,且每个用户可用的inodes的软限制为100,
        硬限制为120。并对磁盘配额情况进行验证测试。(给出相关命令及运行结果)
\end{enumerate}

\section{2.3 题目分析}

\begin{enumerate}
    \item 通过实验一的知识即可实现用户加入同一组,通过chown、chgrp、chmod等指令即
        可更改public的所属人,所属组和权限。
    \item 
\end{enumerate}



