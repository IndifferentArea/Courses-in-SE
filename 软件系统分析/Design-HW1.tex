\documentclass[12pt, a4paper, oneside]{ctexart}
\usepackage{amsmath, amsthm, amssymb, appendix, bm, graphicx, hyperref, mathrsfs, geometry}
\geometry{left=2.54cm, right=2.54cm, top=3.18cm, bottom=3.18cm}

\usepackage{listings}		% 为了避免与页眉的兼容问题可将listings放入table环境中
\lstset{
    basicstyle          =   \sffamily,          % 基本代码风格
    keywordstyle        =   \bfseries,          % 关键字风格
    commentstyle        =   \rmfamily\itshape,  % 注释的风格, 斜体
    stringstyle         =   \ttfamily,  % 字符串风格
    flexiblecolumns,                % 别问为什么, 加上这个
    numbers             =   left,   % 行号的位置在左边
    showspaces          =   false,  % 是否显示空格, 显示了有点乱, 所以不现实了
    numberstyle         =   \zihao{-5}\ttfamily,    % 行号的样式, 小五号, tt等宽字体
    showstringspaces    =   false,
    captionpos          =   t,      % 这段代码的名字所呈现的位置, t指的是top上面
    frame               =   lrtb,   % 显示边框
}

\linespread{1.5}
\renewcommand{\abstractname}{\Large\textbf{摘要}}

\begin{document}

\thispagestyle{empty}

\begin{figure}[t]
    \centering
    \includegraphics[width=13cm]{logo1.png}
\end{figure}

\vspace*{\fill}
    \begin{center}
        \Huge\textbf{\lstinline{Action} \& \lstinline{ActionListener}的实现\\及与其他事件处理机制的比较分析}
    \end{center}
\vspace*{\fill}

\begin{table}[b]
    \centering
    \large
    \begin{tabular}{ll}
    \textbf{课程:} & 软件系统设计与分析 \\
    \textbf{姓名:} & 杨豪 \\
    \textbf{班级:} & 软件2101 \\
    \textbf{时间:} & 2022年9月 \\
    \end{tabular}
\end{table}

\newpage

\thispagestyle{empty}
\begin{abstract}
    这里是摘要. 
    \par\textbf{关键词:}这里是关键词; 这里是关键词. 
\end{abstract}

\newpage
\pagenumbering{Roman}
\setcounter{page}{1}
\tableofcontents
\newpage
\setcounter{page}{1}
\pagenumbering{arabic}

\section{事件与事件处理机制概述}

\subsection{事件与事件处理}

\begin{itemize}
    \item 事件(event): 在软件设计中, 事件是可由软件识别并处理的动作(action)或发生情形(occurrence), 通常来自外部环境, 可以由系统、用户或其他方式生成或触发. 
    \begin{itemize}
        \item 事件的来源可能是用户据通过计算机的外围设备与软件交互(eg.通过键盘输入); 
        \item 软件也可以触发它自己的事件集进入事件循环(eg.完成通信任务)
    \end{itemize}
    \item 事件处理(event processing): 软件识别某一特定事件进行并对其进行特定的处理方式
\end{itemize}

\subsection{事件处理机制及其组成}
一个事件机制一般有三个组成部分, 这里以生活中的事为例: 我一边打游戏一边烧水
\begin{itemize}
    \item 事件源(source): 即事件的发送者. 在上例中为水壶; 
    \item 事件(event): 事件源发出的一种信息或状态. 比如上例的警报声, 它代表着水开了; 
    \item 事件侦听者(listener): 对事件作出反应的对象. 比如上例中打游戏的我
\end{itemize}
在设计事件机制时一般把侦听者设计为一个函数, 当事件发送时, 调用此函数。比如上例中可以把倒水设计为侦听者. 

常见的事件处理机制: 
\begin{itemize}
    \item 
\end{itemize}

\section{\lstinline{Action} \& \lstinline{ActionListener}}








%%% reference
\newpage

\begin{thebibliography}{99}
    \bibitem{a}作者. \emph{文献}[M]. 地点:出版社,年份.
    \bibitem{b}作者. \emph{文献}[M]. 地点:出版社,年份.
\end{thebibliography}


%%% Appendix
\newpage

\begin{appendices}
    \renewcommand{\thesection}{\Alph{section}}
    \section{参考}
        这里是附录. 
\end{appendices}



\end{document}