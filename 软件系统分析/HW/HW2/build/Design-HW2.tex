\documentclass[12pt, a4paper, oneside]{ctexart}
\usepackage{amsmath, amsthm, amssymb, appendix, bm, graphicx, mathrsfs, geometry, xcolor}
\geometry{left=2.54cm, right=2.54cm, top=3.18cm, bottom=3.18cm}
\usepackage[colorlinks, linkcolor=black]{hyperref}

\usepackage{listings}		% 为了避免与页眉的兼容问题可将listings放入table环境中
\lstset{
    basicstyle          =   \sffamily,          % 基本代码风格
    keywordstyle        =   \color{blue},          % 关键字风格
    keywordstyle    =   [2] \color{teal},
    commentstyle        =   \rmfamily\itshape,  % 注释的风格,斜体
    stringstyle         =   \ttfamily,  % 字符串风格
    flexiblecolumns,                % 别问为什么,加上这个
    numbers             =   left,   % 行号的位置在左边
    showspaces          =   false,  % 是否显示空格,显示了有点乱,所以不现实了
    numberstyle         =   \zihao{-5}\ttfamily,    % 行号的样式,小五号,tt等宽字体
    showstringspaces    =   false,
    captionpos          =   t,      % 这段代码的名字所呈现的位置,t指的是top上面
    frame               =   lrtb,   % 显示边框
    basicstyle          =   \zihao{-5}\ttfamily,
    stringstyle         =   \color{magenta},
    commentstyle        =   \color{red}\ttfamily,
    breaklines          =   true,   % 自动换行,建议不要写太长的行
    columns             =   fixed,  % 如果不加这一句,字间距就不固定,很丑,必须加
    basewidth           =   0.5em,
}


\linespread{1.5}
\renewcommand{\abstractname}{\Large\textbf{摘要}}

\begin{document}

\thispagestyle{empty}

\begin{figure}[t]
    \centering
    \includegraphics[width=13cm]{../pic/xjtu.png}
\end{figure}

\vspace*{\fill}
    \begin{center}
        \centering
        \vspace{-3cm}
        \fangsong\huge{本科生课程报告} \\\kaishu \Huge{\textbf{需求管理系统分析与设计}}
    \end{center}
\vspace*{\fill}

\begin{table}[b]
    \centering
    \large
    \begin{tabular}{ll}
        \textbf{课程:} & 软件系统设计与分析 \\
        \textbf{姓名:} & 杨豪 \\
        \textbf{班级:} & 软件2101 \\
        \textbf{时间:} & 2022年10月 \\
    \end{tabular}
\end{table}

\newpage

\thispagestyle{empty}
\begin{abstract}
    事件处理是面向对象分析与设计中最常用的功能之一. 本文先简单论述了事件、事件处理、事件处理机制的概念, 然后以Java语言分别实现了基本的事件处理机制: 
    同步调用、回调函数. 随后着重分析了Action\&ActionListener及其父类EventObject等接口的Java内部实现, 尝试用代码实现了一个简单的ActionListener机制. 
    最后总结了以上三种事件处理机制的优缺点. 
    \par\textbf{关键词:}事件处理机制; Java; 面向对象; Listener. 
\end{abstract}

\newpage
\pagenumbering{Roman}
\setcounter{page}{1}
\tableofcontents
\newpage
\setcounter{page}{1}
\pagenumbering{arabic}

\section{需求管理系统概述}

\subsection{需求工程}

把所有与需求直接相关的活动通称为需求工程。需求工程中的活动可分为两大类,一类属于需求开发,另一类属于需
求管理。

%%% reference
\newpage

\begin{thebibliography}{99}
    \bibitem{a}黄文海.Java多线程编程实战指南(设计模式篇).[M].北京:电子工业出版社,2006.
    \bibitem{b}action与actionlistener的关系.[DB/OL],\newline \url{https://zhuanlan.zhihu.com/p/65192541},2009.8.21.
    \bibitem{c}Java——事件处理机制.[DB/OL],
        \newline \url{https://blog.csdn.net/qq_19865749/article/details/70184968},2017.4.7.
    \bibitem{d}Java ActionListener Interface.[DB/OL].
        \newline \url{https://www.javatpoint.com/java-actionlistener},2021.
    \bibitem{e}Interface ActionListener java SE 17 \& JDK 17 [DB/OL]\newline
        \url{https://docs.oracle.com/en/java/javase/17/docs/api/java.desktop/java/awt/event/ActionListener.html},2022.
\end{thebibliography}


%%% Appendix
\newpage

\begin{appendices}
    \renewcommand{\thesection}{\Alph{section}}
    \section{\LaTeX 排版参考}
        本文中所用\LaTeX 代码参考自
        \begin{itemize}
            \item  \href{https://zhuanlan.zhihu.com/p/385727082}{【LaTeX】自用简洁模板(六):学校作业} 
            \item  \href{https://zhuanlan.zhihu.com/p/65441079}{LaTeX 里「添加程序代码」的完美解决方案}
        \end{itemize}
\end{appendices}



\end{document}