\documentclass[11pt]{article}  % need to compile twice
\usepackage{amsmath, textcomp, amssymb, geometry, graphicx, enumerate, ctex, xcolor, float}
\usepackage[colorlinks, linkcolor=black]{hyperref}
\usepackage{listings}		% 为了避免与页眉的兼容问题可将listings放入table环境中
\lstset{
    basicstyle          =   \sffamily,          % 基本代码风格
    keywordstyle        =   \color{blue},          % 关键字风格
    keywordstyle    =   [2] \color{teal},
    commentstyle        =   \rmfamily\itshape,  % 注释的风格,斜体
    stringstyle         =   \ttfamily,  % 字符串风格
    flexiblecolumns,                % 别问为什么,加上这个
    numbers             =   left,   % 行号的位置在左边
    showspaces          =   false,  % 是否显示空格,显示了有点乱,所以不现实了
    numberstyle         =   \zihao{-5}\ttfamily,    % 行号的样式,小五号,tt等宽字体
    showstringspaces    =   false,
    captionpos          =   t,      % 这段代码的名字所呈现的位置,t指的是top上面
    frame               =   lrtb,   % 显示边框
    basicstyle          =   \zihao{-4}\ttfamily,
    stringstyle         =   \color{magenta},
    commentstyle        =   \color{red}\ttfamily,
    breaklines          =   true,   % 自动换行,建议不要写太长的行
    columns             =   fixed,  % 如果不加这一句,字间距就不固定,很丑,必须加
    basewidth           =   0.5em,
}
\geometry{left=2.54cm, right=2.54cm, top=3.18cm, bottom=3.18cm}

\def\Name{杨豪\space}  % Your name
\def\SID{2206213297}  % Your student ID number

% need to be confirmed before each time writing and committing 
\def\Homework{2} % Number of Homework
\def\Session{2022-Fall}
\def\CourseCodeName{SOFT500227: Computer Graphics}
\def\simCourseName{CG}

\title{\vspace{-4cm}\CourseCodeName \space
        \Session \protect\\  Homework-\textbf{\Homework} Solutions}
\author{软件2101 \Name \space 学号: \SID}
\markright{\simCourseName\ \space \Session\  HW-\Homework\ \Name}
\date{\today}

\begin{document}

\maketitle

\textbf{Honor Code: I promise that I finished the homework solutions on my own without copying other people's 
    work.}

\section*{第2章 光栅图形学}

\subsection*{2.1 直线算法的实现与分析}

\subsubsection*{DDA算法}

\begin{lstlisting}[language = C]
// DDA算法中需要大量用到 浮点数运算 和 分支判断,会大大减缓硬件速度
void DDALine(int x0, int y0, int x1, int y1, Color color)
{		
	float dx = x1 - x0, dy = y1 - y0;
    float k = dy / dx, y = y0;
    if(k <= 1){
        float y = y0;
        for(int x = x0; x <= x1; x ++){
            draw_pixel(x, (int)(y + 0.5), color);
	        y += k;
        }
    }else{
        float x = x0;
        for(int y = y0; y <= y1; y ++){
            draw_pixel(x, (int)(y + 0.5), color);		   
	        x += 1 / k;
        }
    }
}
\end{lstlisting}

\subsubsection*{中点画线法}

\begin{lstlisting}[language = C]
// 中点画线法在增量算法的优化下速度大幅提升,
// 且通过优化d0让算法完全在整数运算中实现,大大提高了运算速度
void MidpointLine(int x0, int y0, int x1, int y1, Color color) {
    int a = y0 - y1, b = x1 - x0, x = x0, y = y0;
    int d0 = 2 * a + b, d1 = 2 * a, d2 = 2 * (a + b);
    while(x <= x1) {
        draw_pixel(x, y, color);
        if(d0 < 0){
            x ++, y ++, d += d2;  //右上方
        }else{
            x ++, d += d1;       //右方
        }
    }
}
\end{lstlisting}

\subsubsection*{Bresenham算法}

\begin{lstlisting}[language = C]
// 比起中点画线法,Bresenham算法的推导过程简单易理解;
// 且由于判断都是定性,同样可以通过优化初值让算法完全由整数运算构成
void BresenhamLine(int x0, int y0, int x1, int y1, int color){
    int dx = x1 - x0, dy = y1 - y0, x = x0, y = y0;
    int e = -dx;
    for(int i = 0; i <= dx; i++){
        draw_pixel(x, y, color);
        x ++, e += 2 * dy;
        if(e >= 0){
            y ++, e -= 2 * dx;
        }
    }
}
\end{lstlisting}

\subsection*{2.2 中点画圆法}

\begin{lstlisting}[language = C]
// sym_draw(int x, int y, Color color)函数依照八分对称性画出(x,y)的对称点
void sym_draw(int x, int y, Color color) {
	draw_pixel(x, y, color);
	draw_pixel(y, x, color);
	draw_pixel(-x, y, color);
	draw_pixel(-y, x, color);
	draw_pixel(x, -y, color);
	draw_pixel(y, -x, color);
	draw_pixel(-x, -y, color);
	draw_pixel(-y, -x, color);
}

void MidPointCircle(int r, Color color){
    int x, y, d;
    int x = 0, y = r, e = 1 - r;            
    sym_draw(x, y, color);             
    while (x <= y) {                        // 到直线x=y结束
        if(e < 0) e += 2 * x + 3;           // 右侧
        else e += 2 * (x - y) + 5, y--;     // 右下
        sym_draw(x ++, y, color);
    }
}
\end{lstlisting}

\section*{Other things}

\LaTeX \space code refer to these things and was complied on texlive2020.
\begin{itemize}
    \item  \href{https://www.eecs70.org/assets/misc/homework_template.tex}{UCB-CS70's given homework template.} 
    \item  \href{https://www.latexlive.com}{A free website useful to edit \LaTeX \space formula code.}
\end{itemize}

The purpose of writing in English is to adapt to bilingual teaching and to improve my poor English 
writing skills in preparation for a possible future exchange program. 

Thanks for your correcting and grading :).

\end{document}

 
