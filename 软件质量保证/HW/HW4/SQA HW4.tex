\documentclass[11pt, oneside]{article}  % need to compile twice
\usepackage{amsmath, textcomp, amssymb, geometry, graphicx, enumerate, ctex, fancyhdr, float}
\usepackage[colorlinks, linkcolor=black]{hyperref}
\usepackage{multirow}
\usepackage[normalem]{ulem}
\usepackage{subcaption}
\useunder{\uline}{\ul}{}

\usepackage{listings}		% 为了避免与页眉的兼容问题可将listings放入table环境中
\lstset{
    basicstyle          =   \sffamily,          % 基本代码风格
    keywordstyle        =   \bfseries,          % 关键字风格
    commentstyle        =   \rmfamily\itshape,  % 注释的风格, 斜体
    stringstyle         =   \ttfamily,  % 字符串风格
    flexiblecolumns,                % 别问为什么, 加上这个
    numbers             =   left,   % 行号的位置在左边
    showspaces          =   false,  % 是否显示空格, 显示了有点乱, 所以不现实了
    numberstyle         =   \zihao{-5}\ttfamily,    % 行号的样式, 小五号, tt等宽字体
    showstringspaces    =   false,
    captionpos          =   t,      % 这段代码的名字所呈现的位置, t指的是top上面
    frame               =   lrtb,   % 显示边框
}

\geometry{left=2.54cm, right=2.54cm, top=3.18cm, bottom=3.18cm}

\def\Name{杨豪\space}  % Your name
\def\SID{2206213297}  % Your student ID number

% need to be confirmed before each time writing and committing 
\def\Homework{4} % Number of Homework
\def\Session{2022-Fall}
\def\CourseCodeName{SOFT410911: Software Quality Assurance}
\def\simCourseName{SQA}

\title{\vspace{-4cm}\CourseCodeName \space
        \Session \protect\\  Homework-\textbf{\Homework} Solutions}
\author{软件2101 \Name \space 学号: \SID}
\date{\today}

\markright{\simCourseName\ \space \Session\  HW-\Homework\ \Name}

\begin{document}

\maketitle

\textbf{Honor Code: I promise that I finished the homework solutions on my own without copying other people's work.}

\section*{Risk-based Testing}

采用之前使用过的QQ在web端的注册功能为例:
\subsection*{1. 风险识别}
可能的风险有
\begin{enumerate}[a.]
    \item 同时注册的人过多,服务器崩溃;
    \item 不同的浏览器/设备端可能会导致网页上的错误;
    \item 新增输入框或修改输入的限制可能会导致问题;
    \item 非法输入导致的错误;
    \item 点击页面上的外部链接可能导致的跳转失败
\end{enumerate}

\subsection*{2. 风险分析}
综合查阅资料和模拟不同的用户体验,分析风险等级如下:
\begin{table}[H]
    \centering
    \begin{tabular}{|c|c|c|c|}
        \hline
        风险点 & 风险概率(1-5) & 风险影响(1-5) & 风险等级 \\ \hline
        a   & 1         & 5         & 5    \\ \hline
        b   & 3         & 3         & 9    \\ \hline
        c   & 2         & 2         & 4    \\ \hline
        d   & 5         & 4         & 20   \\ \hline
        e   & 4         & 1         & 4    \\ \hline
    \end{tabular}
\end{table}

\subsection*{3. 测试用例}

由2 最高的两个等级为
\begin{itemize}
    \item[d.] 非法输入导致的错误:可以通过HW-3提到的\textbf{等价性测试}加上\textbf{边界测试}可以避免这种问题。
    \item[b.] 不同的浏览器/设备端可能会导致网页上的错误:对所有常用的硬件设备端和软件设备端排列组合,测试每一种可能以确定注册功能无误。举例如下
    \begin{itemize}
        \item 平台:PC、手机、其他(如嵌入式);
        \item 操作系统:Windows、MacOS、Android、iOS、Linux;
        \item 浏览器内核:IE内核、Chromium内核、Firefox内核、Safari内核、Opera浏览器内核;
    \end{itemize}
    共75种可能。
\end{itemize}

\subsection*{4. 具体执行测试用例}

\begin{itemize}
    \item[d.] 参见HW-3;
    \item[b.] 只需对部署出的网页在不同的环境下进入即可,因选用QQ这一成熟的软件系统以及本人设备数不够未能顺利测试。
\end{itemize}


\section*{Other things}

    \LaTeX \space code refer to these things and was complied on texlive2020 by \lstinline{xelatex}.
    \begin{itemize}
        \item  \href{https://www.eecs70.org/assets/misc/homework_template.tex}{UCB-CS70's given homework template.} 
        \item  \href{https://www.tablesgenerator.com/}{A free website useful to edit \LaTeX \space table code.}
    \end{itemize}

    Thanks for your correcting and grading :).

\end{document}

 
